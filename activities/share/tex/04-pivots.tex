\documentclass[12pt]{article}

\pagestyle{empty}
\setlength{\topmargin}{0in}
\setlength{\headheight}{0in}
\setlength{\topsep}{0in}
\setlength{\textheight}{9in}
\setlength{\oddsidemargin}{0in}
\setlength{\evensidemargin}{0in}
\setlength{\textwidth}{6.5in}

\usepackage{palatino,graphics,amsmath}

\newcommand{\real}{{\bf R}}
\newcommand{\ds}{\displaystyle}
\newcommand{\vs}[1]{\vspace{#1in}}
\renewcommand{\vss}[1]{\vspace*{#1in}}
\begin{document}

\noindent
{\bf Mathematics 227} \\ 
{\bf Pivot positions}

\bigskip
\begin{enumerate}
\item Shown below are three augmented matrices in reduced row echelon
  form.  For each matrix, identify the pivot positions and describe
  the solution space.

  $
  \left[
    \begin{array}{rrr|r}
      1 & 0 & 0 & 3 \\
      0 & 1 & 0 & 0 \\
      0 & 0 & 1 & -2 \\
      0 & 0 & 0 & 0 \\
    \end{array}
  \right]
  $

  \bigskip
  $
  \left[
    \begin{array}{rrr|r}
      1 & 0 & 2 & 3 \\
      0 & 1 & -1 & 0 \\
      0 & 0 & 0 & 0 \\
      0 & 0 & 0 & 0 \\
    \end{array}
  \right]
  $

  \bigskip
  $
  \left[
    \begin{array}{rrr|r}
      1 & 0 & 2 & 0 \\
      0 & 1 & -1 & 0 \\
      0 & 0 & 0 & 1 \\
      0 & 0 & 0 & 0 \\
    \end{array}
  \right]
  $

  Each of these augmented matrices above has a row in
  which each entry is zero.  What, if anything, does the
  presence of such a row tell us about the solution space to the
  linear system?

  \vs{1}
  Give an example of a $3\times5$ augmented matrix
  in reduced row echelon form that represents a consistent
  system.  Indicate the pivot positions in your matrix and
  explain why these pivot positions guarantee a consistent
  system.

  \vs{1.5}

  \newpage
  Give an example of a $3\times5$ augmented matrix in
  reduced row echelon form that represents an inconsistent
  system.  Indicate the pivot positions in your matrix and
  explain why these pivot positions guarantee an inconsistent
  system.

  \vs{1.5}
  How do the pivot positions determine whether a linear system is
  consistent or not?

  \vs{1.1}
  Suppose we have a linear system for
  which the {\em coefficient} matrix has the following
  reduced row echelon form.
  $$
  \left[
    \begin{array}{rrrrr}
      1 & 0 & 0 & 0 & -1 \\
      0 & 1 & 0 & 0 & 2  \\
      0 & 0 & 1 & 0 & 0  \\
      0 & 0 & 0 & 1 & -3 \\
    \end{array}
  \right]
  $$
  Even though we don't have any information about the right-hand side
  of the equations, what can you say about the consistency of the
  linear system?  Explain your thinking.

  \vs{1.1}

  The following augmented matrix is in reduced row echelon form.
  Describe the solution space;  in particular, identify any free and
  basic variables.

  $
  \left[
    \begin{array}{rrrrr|r}
      1 & -2 & 0 & 4 & 0 & 2 \\
      0 & 0 & 1 & 3 & 0 & -1 \\
      0 & 0 & 0 & 0 & 1 & 4 \\
      0 & 0 & 0 & 0 & 0 & 0 \\
    \end{array}
  \right]
  $

  How can we identify the free and basic variables by looking at the
  pivot positions?

  \vs{1.1}
  \newpage
  If possible, give an example of a $3\times5$
  augmented matrix that corresponds to a system of linear equations
  having a unique solution. If it is not possible, explain why.

  \vs{1.1}

  If possible, give an example of a $5\times3$
  augmented matrix that corresponds to a system of linear equations
  having a unique solution. If it is not possible, explain why.

  \vs{1}

  What condition on the pivot positions guarantees that a system of
  linear equations has a unique solution?

  \vs{1}

  If a system of linear equations has a unique solution, what can we
  say about the relationship between the number of equations and the
  number of unknowns?  Use the concept of pivot positions to explain
  your thinking.


\end{enumerate}


\end{document}
