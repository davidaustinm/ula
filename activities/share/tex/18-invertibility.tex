\documentclass[12pt]{article}

\pagestyle{empty}
\setlength{\topmargin}{0in}
\setlength{\headheight}{0in}
\setlength{\topsep}{0in}
\setlength{\textheight}{9in}
\setlength{\oddsidemargin}{0in}
\setlength{\evensidemargin}{0in}
\setlength{\textwidth}{6.5in}

\usepackage{palatino,graphics,amsmath,amssymb,enumitem}

\newcommand{\ds}{\displaystyle}
\newcommand{\vs}[1]{\vspace{#1in}}
\renewcommand{\vss}[1]{\vspace*{#1in}}
\newcommand{\bvec}{{\mathbf b}}
\newcommand{\cvec}{{\mathbf c}}
\newcommand{\dvec}{{\mathbf d}}
\newcommand{\evec}{{\mathbf e}}
\newcommand{\fvec}{{\mathbf f}}
\newcommand{\qvec}{{\mathbf q}}
\newcommand{\uvec}{{\mathbf u}}
\newcommand{\vvec}{{\mathbf v}}
\newcommand{\wvec}{{\mathbf w}}
\newcommand{\xvec}{{\mathbf x}}
\newcommand{\yvec}{{\mathbf y}}
\newcommand{\zvec}{{\mathbf y}}
\newcommand{\zerovec}{{\mathbf 0}}
\newcommand{\real}{{\mathbb R}}
\newcommand{\twovec}[2]{\left[\begin{array}{r}#1 \\ #2
    \end{array}\right]}
\newcommand{\ctwovec}[2]{\left[\begin{array}{c}#1 \\ #2
   \end{array}\right]}
\newcommand{\threevec}[3]{\left[\begin{array}{r}#1 \\ #2 \\ #3
  \end{array}\right]}
\newcommand{\cthreevec}[3]{\left[\begin{array}{c}#1 \\ #2 \\ #3
    \end{array}\right]}
\newcommand{\fourvec}[4]{\left[\begin{array}{r}#1 \\ #2 \\ #3 \\ #4
    \end{array}\right]}
\newcommand{\cfourvec}[4]{\left[\begin{array}{c}#1 \\ #2 \\ #3 \\ #4
    \end{array}\right]}
\newcommand{\mattwo}[4]{\left[\begin{array}{rr}#1 & #2 \\ #3 & #4 \\ \end{array}\right]}
\renewcommand{\span}[1]{\text{Span}\{#1\}}
\newcommand{\bcal}{{\cal B}}
\newcommand{\ccal}{{\cal C}}
\newcommand{\scal}{{\cal S}}
\newcommand{\wcal}{{\cal W}}
\newcommand{\ecal}{{\cal E}}
\newcommand{\coords}[2]{\left\{#1\right\}_{#2}}
\newcommand{\gray}[1]{\color{gray}{#1}}
\newcommand{\lgray}[1]{\color{lightgray}{#1}}
\newcommand{\rank}{\text{rank}}
\newcommand{\col}{\text{Col}}
\newcommand{\nul}{\text{Nul}}

\begin{document}

\noindent
{\bf Mathematics 227} \\ 
{\bf Invertible matrices}

\bigskip
\begin{enumerate}
\item Consider the matrix
  $$
  A =
  \left[
    \begin{array}{ccc}
      1 & 3 & -2 \\
      0 & 1 & 2 \\
      1 & -1 & 2 \\
    \end{array}
  \right].
  $$

  Explain why the matrix $A$ is invertible.

  \vs{1}
  Find the inverse $A^{-1}$ by 
  row reducing $\left[\begin{array}{c|c}A&I\end{array}\right]$.

  Note:  In Sage, you may create the $3\times3$ identity matrix by
  {\tt identity\_matrix(3)}.  You may augment
  $A$ using {\tt A.augment(identity\_matrix(3))}


  \vs{1.5}
  Use the inverse $A^{-1}$ to solve the equation
  $$
  A\xvec = \threevec{-4}38.
  $$
  
  Note: In Sage, you may find the inverse of a matrix 
  {\tt B} using {\tt B.inverse()}.

  \vs{1}
\item Suppose that $A$ is an invertible $5\times5$ matrix.  What can
  you guarantee about the span of the columns of $A$?

  \vs{1}
  \newpage
  What can you guarantee about the linear independence of the columns
  of $A$?

  \vs{1}
\item If $A$ is a $7\times7$ matrix whose columns are linearly
  independent, can you guarantee that $A$ is invertible?  Explain your
  thinking.  

  \vs{1}
\item Suppose that both $A$ and $B$ are invertible $n\times n$
  matrices.  Explain why $(AB)^{-1} = B^{-1}A^{-1}$ by multiplying
  this matrix by $AB$.

  \vs{1}
  Explain why the product of two invertible $n\times n$ matrices is
  invertible. 

  \vs{1}
\item We will now look at a special type of matrix having the form
  $$
  L =
  \left[
    \begin{array}{ccc}
      * & 0 & 0 \\
      * & * & 0 \\
      * & * & * \\
    \end{array}
  \right],\hspace*{24pt}
  U =
  \left[
    \begin{array}{ccc}
      * & * & * \\
      0 & * & * \\
      0 & 0 & * \\
    \end{array}
  \right].
  $$
  The matrix $L$ is called {\em lower triangular} because all the
  entries above the diagonal are zero;  in the same way, $U$ is called
  {\em upper triangular}.

  Consider the following two lower triangular matrices
  $$
  L_1 =
  \left[
    \begin{array}{ccc}
      1 & 0 & 0 \\
      2 & 2 & 0 \\
      -1 & 3 & 1 \\
    \end{array}
  \right],\hspace*{24pt}
  L_1 =
  \left[
    \begin{array}{ccc}
      1 & 0 & 0 \\
      2 & 0 & 0 \\
      -1 & 3 & 1 \\
    \end{array}
  \right].
  $$
  \newpage
  Imagine where the pivot positions in these matrices occur.  Which
  one of them is invertible and which is not invertible?

  \vs{1}
  What condition on the diagonal entries of a triangular matrix
  determines whether the matrix is invertible or not?

  \vs{1}
\item We will now revisit Gaussian elimination, the algorithm we
  learned at the beginning of the class that we have been using to row
  reduce matrices.

  Consider the matrix
  $
  A = 
  \left[
    \begin{array}{ccc}
      1 & 2 & 1 \\
      2 & 0 & -2 \\
      -1 & 2 & -1 \\
    \end{array}
  \right]
  $.
  Perform a row operation to make the first
  entry in the second row zero.

  \vs{1}
  If
  $L_1 =
  \left[
    \begin{array}{ccc}
      1 & 0 & 0 \\
      -2 & 1 & 0 \\
      0 & 0 & 1 \\
    \end{array}
  \right]
  $, verify that $L_1A$ is the same matrix you obtained from this row
  operation.

  \vs{1}
  Explain how you know that  $L_1$ is invertible.

  \vs{1}
  \newpage
  Find $L_1^{-1}$ and explain what row operation it performs.  

  \vs{1}
  Suppose that
  $L_2 =
  \left[
    \begin{array}{ccc}
      1 & 0 & 0 \\
      0 & 11 & 0 \\
      0 & 0 & 1 \\
    \end{array}
  \right]
  $.  Find $L_2A$ and explain what row operation it performs.

  \vs{1}
  What is $L_2^{-1}$ and explain what row operation it performs.

  \vs{1}
  Suppose that
  $
  P =
  \left[
    \begin{array}{ccc}
      0 & 0 & 1 \\
      0 & 1 & 0 \\
      1 & 0 & 0 \\
    \end{array}
  \right]
  $.  Find $PA$ and explain what row operation it performs.
  
  \vs{1}
  What is $P^{-1}$ and explain what row operation it performs.
  
  
  
              
    
    


\end{enumerate}


\end{document}
