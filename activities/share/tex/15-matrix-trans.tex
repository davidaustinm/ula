\documentclass[12pt]{article}

\pagestyle{empty}
\setlength{\topmargin}{0in}
\setlength{\headheight}{0in}
\setlength{\topsep}{0in}
\setlength{\textheight}{9in}
\setlength{\oddsidemargin}{0in}
\setlength{\evensidemargin}{0in}
\setlength{\textwidth}{6.5in}

\usepackage{palatino,graphics,amsmath,amssymb,enumitem}

\newcommand{\ds}{\displaystyle}
\newcommand{\vs}[1]{\vspace{#1in}}
\renewcommand{\vss}[1]{\vspace*{#1in}}
\newcommand{\bvec}{{\mathbf b}}
\newcommand{\cvec}{{\mathbf c}}
\newcommand{\dvec}{{\mathbf d}}
\newcommand{\evec}{{\mathbf e}}
\newcommand{\fvec}{{\mathbf f}}
\newcommand{\qvec}{{\mathbf q}}
\newcommand{\uvec}{{\mathbf u}}
\newcommand{\vvec}{{\mathbf v}}
\newcommand{\wvec}{{\mathbf w}}
\newcommand{\xvec}{{\mathbf x}}
\newcommand{\yvec}{{\mathbf y}}
\newcommand{\zvec}{{\mathbf y}}
\newcommand{\zerovec}{{\mathbf 0}}
\newcommand{\real}{{\mathbb R}}
\newcommand{\twovec}[2]{\left[\begin{array}{r}#1 \\ #2
    \end{array}\right]}
\newcommand{\ctwovec}[2]{\left[\begin{array}{c}#1 \\ #2
   \end{array}\right]}
\newcommand{\threevec}[3]{\left[\begin{array}{r}#1 \\ #2 \\ #3
  \end{array}\right]}
\newcommand{\cthreevec}[3]{\left[\begin{array}{c}#1 \\ #2 \\ #3
    \end{array}\right]}
\newcommand{\fourvec}[4]{\left[\begin{array}{r}#1 \\ #2 \\ #3 \\ #4
    \end{array}\right]}
\newcommand{\cfourvec}[4]{\left[\begin{array}{c}#1 \\ #2 \\ #3 \\ #4
    \end{array}\right]}
\newcommand{\mattwo}[4]{\left[\begin{array}{rr}#1 \amp #2 \\ #3 \amp #4 \\ \end{array}\right]}
\renewcommand{\span}[1]{\text{Span}\{#1\}}
\newcommand{\bcal}{{\cal B}}
\newcommand{\ccal}{{\cal C}}
\newcommand{\scal}{{\cal S}}
\newcommand{\wcal}{{\cal W}}
\newcommand{\ecal}{{\cal E}}
\newcommand{\coords}[2]{\left\{#1\right\}_{#2}}
\newcommand{\gray}[1]{\color{gray}{#1}}
\newcommand{\lgray}[1]{\color{lightgray}{#1}}
\newcommand{\rank}{\text{rank}}
\newcommand{\col}{\text{Col}}
\newcommand{\nul}{\text{Nul}}

\begin{document}

\noindent
{\bf Mathematics 227} \\ 
{\bf Matrix transformations}

\bigskip
\begin{enumerate}
\item Suppose that the matrix transformation $T(\xvec) = A\xvec$ where 
  $
  A =
  \left[
    \begin{array}{cccc}
      3 & 3 & -2 & 1 \\
      0 & 2 & 1 & -3 \\
      -2 & 1 & 4 & -4 \\
    \end{array}
  \right].
  $

  \begin{enumerate}[label=(\alph*)]
  \item what is the dimension of the vectors $\xvec$ that are input
    into $T$?  What is the dimension of the vectors $T(\xvec)$ that
    are output from $T$?
    
    \vs{1}
  \item Describe all the vectors for which $T(\xvec) = \zerovec$.
    
    \vs{1}
  \end{enumerate}

\item Suppose that we work for a company that produces baked goods,
  including cakes, donuts, and eclairs. Our company operates two
  plants, Plant 1 and Plant 2. In one hour of operation,
  \begin{itemize}
  \item Plant 1 produces 10 cakes, 50 donuts, and 30 eclairs.
  \item Plant 2 produces 20 cakes, 30 donuts, and 30 eclairs.
  \end{itemize}
  
  \begin{enumerate}[label=(\alph*)]
    \item If plant 1 operates for $x_1$ hours and plant 2 for $x_2$
      hours, how many cakes $C$ do they produce in total?  How many
      donuts $D$?  How many eclairs $E$?

      \vs{1}
    \item We define a matrix transformation
      $
      T\left(\twovec{x_1}{x_2}\right) = \threevec CDE
      $
      where $\threevec CDE$ represents the quantities of baked goods
      produced when the plants run for $\twovec{x_1}{x_2}$ hours.  If
      $T(\xvec) = A\xvec$, what are the dimensions of $A$?

      \vs{1}
    \item Find the vector $T\left(\twovec 10\right)$ and the vector
      $T\left(\twovec 01\right)$.  Then write the matrix $A$.

      \vs{1}

    \item If we operate plant 1 for 40 hours and plant 2 for 50 hours,
      how many baked goods have we produced?

      \vs{1}
    \item The marketing department says we need to produce
      1500 cakes, 4700 donuts, and 3300 eclairs.  Is it possible to
      meet this order?  If so, how long should the two plants operate?

      \vs{1}
    \item Suppose that
      \begin{itemize}
      \item Each cake requires 4 units of flour and 2 units of
        sugar.
      \item Each donut requires 1 unit of flour and 1 unit of sugar.
      \item Each eclair requires 1 unit of flour and 2 units of sugar.
      \end{itemize}
      Suppose we make $C$ cakes, $D$ donuts, and $E$ eclairs.  How
      many units $F$ of flour and units $S$ of sugar are required?

      \vs{1}
    \item Write a matrix $B$ that defines the matrix transformation
      $R\left(\threevec CDE\right) = \twovec FS$.

      \vs{1}

      \newpage
    \item If plant 1 operates for 30 hours and plant 2 for 20 hours,
      how many units of flour and sugar are required?

      \vs{1}
    \item Find the matrix that represents the transformation
      $P\left(\twovec{x_1}{x_2}\right) = \twovec FS$ that represents the
      ingredients needed when the plants are operated for
      $\twovec{x_1}{x_2}$ hours.

      \vs{1}
    \end{enumerate}

  \item Many geometric operations can be described as matrix
    transformations.  Suppose, for instance, that we want to describe
    the geometric transformation $T$ that rotates vectors in the plane
    counterclockwise by $90^\circ$.  Then $T(\xvec) = A\xvec$ where
    $A$ is a $2\times2$ matrix.

    What is $T\left(\twovec10\right)$ and $T\left(\twovec01\right)$?
    Use these vectors to find the matrix $A$.

    \vs{1}
    What vector results when we rotate $\twovec{-4}{5}$
    counterclockwise by $90^\circ$?

    \vs{1}
  \item Find the $2\times2$ matrix that reflects vectors across the
    vertical axis.  What is the result of reflecting $\twovec{-4}{5}$
    across the vertical axis?

\end{enumerate}


\end{document}
