\documentclass[12pt]{article}

\pagestyle{empty}
\setlength{\topmargin}{0in}
\setlength{\headheight}{0in}
\setlength{\topsep}{0in}
\setlength{\textheight}{9in}
\setlength{\oddsidemargin}{0in}
\setlength{\evensidemargin}{0in}
\setlength{\textwidth}{6.5in}

\usepackage{palatino,graphics,amsmath}

\newcommand{\real}{{\bf R}}
\newcommand{\ds}{\displaystyle}
\newcommand{\vs}[1]{\vspace{#1in}}
\renewcommand{\vss}[1]{\vspace*{#1in}}
\begin{document}

\noindent
{\bf Mathematics 227} \\ 
{\bf Pivot positions}

\bigskip
\begin{enumerate}
\item Shown below are some matrices in reduced row echelon form.
  Identify the pivot positions and state whether the linear system is
  consistent or inconsistent.  If the linear system is consistent,
  imagine the variables are $x_1, x_2,\ldots, x_n$ and identify the
  free and basic variables.

  \bigskip
  $
  \left[
    \begin{array}{cccc|c}
      1 & 3 & 0 & 0 & -2 \\
      0 & 0 & 1 & 0 & -1 \\
      0 & 0 & 0 & 1 & 3 \\
    \end{array}
  \right]
  $

  \bigskip
  $
  \left[
    \begin{array}{cccc|c}
      1 & 3 & 0 & -9 & 0 \\
      0 & 0 & 1 & 3 & 0 \\
      0 & 0 & 0 & 0 & 1 \\
    \end{array}
  \right]
  $

  \bigskip
  $
  \left[
    \begin{array}{cccc|c}
      0 & 1 & 0 & 0 & -2 \\
      0 & 0 & 1 & -2 & -1 \\
      0 & 0 & 0 & 0 & 0 \\
    \end{array}
  \right]
  $

  \bigskip
\item A linear system is called {\em homogeneous} if the right hand
  side of every equation is zero.  The augmented matrix would look
  like this:
  $$
  \left[
    \begin{array}{cccc|c}
      * & * & * & * & 0 \\
      * & * & * & * & 0 \\
      * & * & * & * & 0 \\
    \end{array}
  \right]
  $$

  Use pivot positions to explain why a homogeneous linear system is
  always consistent.

  \vs{1}
  What values for the variables are guaranteed to give a solution?

  \vs{1}

  \newpage
\item Give an example of an augmented matrix representing a linear
  system of 4 equations in 3 unknowns that has a unique solution.

  \vs{1.25}
  Give an example of an augmented matrix representing a linear system
  of 4 equations in 3 unknowns that has infinitely many solutions.

  \vs{1.25}

\item Suppose a coefficient matrix represents a consistent linear
  system no matter what is on the right-hand side of the equations.
  What can you say about the pivot positions?

  \vs{1.5}
  For the coefficient matrix of the last problem, suppose that the
  solution to the corresponding homogeneous equation is unique.  What
  can you say about the pivot positions?

  \vs{1}
  What does this say about the dimensions of the coefficient matrix?

  \vs{1}

  \newpage
\item If a linear system with 26 variables has a unique solution, what
  can 
  you say about the number of equations?  Use pivot positions to
  explain your response.

  \vs{1.5}

\item Suppose that you have a homogeneous linear system with 10
  equations and 10 unknowns and that there is a unique solution.  What
  can you say about the pivot positions?

  \vs{1.25}
  If you change the right-hand side of the equations, can you
  guarantee that the system is consistent?  If so, can you guarantee the
  solution is unique.

  \vs{1.25}

\item Suppose you have a linear system whose $3\times4$ coefficient
  matrix has three pivot columns.  What, if anything, can you say
  about the 
  questions of existence and uniqueness?

  \vs{1}

  \newpage
\item For what values of the parameters $k$ and $l$ is the following
  system consistent?  For which value of $k$ and $l$ is there a unique
  solution?  Justify your responses by identifying the pivot positions.

  $$
  \left[
    \begin{array}{cc|c}
      -1 & k & l \\
      2 & 4 & 3 \\
    \end{array}
  \right]
  $$

\end{enumerate}


\end{document}
