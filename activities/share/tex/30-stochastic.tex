\documentclass[12pt]{article}

\pagestyle{empty}
\setlength{\topmargin}{0in}
\setlength{\headheight}{0in}
\setlength{\topsep}{0in}
\setlength{\textheight}{9in}
\setlength{\oddsidemargin}{0in}
\setlength{\evensidemargin}{0in}
\setlength{\textwidth}{6.5in}

\usepackage{palatino,graphics,amsmath,amssymb,enumitem}

\newcommand{\ds}{\displaystyle}
\newcommand{\vs}[1]{\vspace{#1in}}
\renewcommand{\vss}[1]{\vspace*{#1in}}
\newcommand{\bvec}{{\mathbf b}}
\newcommand{\cvec}{{\mathbf c}}
\newcommand{\dvec}{{\mathbf d}}
\newcommand{\evec}{{\mathbf e}}
\newcommand{\fvec}{{\mathbf f}}
\newcommand{\qvec}{{\mathbf q}}
\newcommand{\uvec}{{\mathbf u}}
\newcommand{\vvec}{{\mathbf v}}
\newcommand{\wvec}{{\mathbf w}}
\newcommand{\xvec}{{\mathbf x}}
\newcommand{\yvec}{{\mathbf y}}
\newcommand{\zvec}{{\mathbf y}}
\newcommand{\zerovec}{{\mathbf 0}}
\newcommand{\real}{{\mathbb R}}
\newcommand{\twovec}[2]{\left[\begin{array}{r}#1 \\ #2
    \end{array}\right]}
\newcommand{\ctwovec}[2]{\left[\begin{array}{c}#1 \\ #2
   \end{array}\right]}
\newcommand{\threevec}[3]{\left[\begin{array}{r}#1 \\ #2 \\ #3
  \end{array}\right]}
\newcommand{\cthreevec}[3]{\left[\begin{array}{c}#1 \\ #2 \\ #3
    \end{array}\right]}
\newcommand{\fourvec}[4]{\left[\begin{array}{r}#1 \\ #2 \\ #3 \\ #4
    \end{array}\right]}
\newcommand{\cfourvec}[4]{\left[\begin{array}{c}#1 \\ #2 \\ #3 \\ #4
    \end{array}\right]}
\newcommand{\mattwo}[4]{\left[\begin{array}{rr}#1 & #2 \\ #3 & #4 \\ \end{array}\right]}
\renewcommand{\span}[1]{\text{Span}\{#1\}}
\newcommand{\bcal}{{\cal B}}
\newcommand{\ccal}{{\cal C}}
\newcommand{\scal}{{\cal S}}
\newcommand{\wcal}{{\cal W}}
\newcommand{\ecal}{{\cal E}}
\newcommand{\coords}[2]{\left\{#1\right\}_{#2}}
\newcommand{\gray}[1]{\color{gray}{#1}}
\newcommand{\lgray}[1]{\color{lightgray}{#1}}
\newcommand{\rank}{\text{rank}}
\newcommand{\col}{\text{Col}}
\newcommand{\nul}{\text{Nul}}

\begin{document}

\noindent
{\bf Mathematics 227} \\ 
{\bf Stochastic matrices and Markov chains}

\bigskip A {\em probability vector} is a vector whose entries are
non-negative and add to 1.  For instance,
$
\vvec = \twovec{0.7}{0.3}
$
is a probability vector.  A {\em stochastic} matrix is a matrix whose
columns are all probability vectors.  For instance
$
A = \mattwo{0.7}{0.6}{0.3}{0.4}
$
is a stochastic matrix.  

\begin{enumerate}
\item Find the product $A\vvec$ and verify that it is a probability
  vector.  

  \vs{1}
  It is always the case that we obtain a probability vector when we
  multiply a stochastic matrix by a probability vector.

  \medskip
  Find the eigenvalues of $A$ and a basis for each eigenspace.

  \vs{2}
  Find a probability vector $\qvec$ in $E_1$.  To do this,
  remember that 
  every vector in $E_1$ is a scalar multiple of the basis vector.
  What scalar multiple of your basis vector gives a probability vector?

  \vs{1}
  Write $\xvec_0=\twovec10$ as a linear combination of
  eigenvectors of $A$.

  \vs{1}
  \newpage
  Now consider the dynamical system $\xvec_{k+1} = A\xvec_k$.
  What happens to $\xvec_k$ as $k$ grows increasingly large?  

  \vs{1}
  How is the result related to the probability vector $\qvec$ that you
  found in $E_1$?

  \vs{1}
  Rather than $\xvec_0=\twovec01$, suppose you began with another
  initial state vector $\xvec_0$.  What happens to $\xvec_k$ as $k$
  grows increasingly large?

  \vs{1}
\item Now consider the stochastic matrix
$A = 
\left[
  \begin{array}{cc}
    0 & 1 \\
    1 & 0 \\
  \end{array}
\right]
$.  Find the eigenvalues of $A$ and a basis for each eigenspace.

\vs{1.5}
Write the initial state vector $\xvec_0 = \twovec10$ as a linear
combination of eigenvectors of $A$.

\vs{1}
What happens to $\xvec_k$ as $k$ grows increasingly large?

\end{enumerate}


\end{document}
