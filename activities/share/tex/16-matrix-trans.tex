\documentclass[12pt]{article}

\pagestyle{empty}
\setlength{\topmargin}{0in}
\setlength{\headheight}{0in}
\setlength{\topsep}{0in}
\setlength{\textheight}{9in}
\setlength{\oddsidemargin}{0in}
\setlength{\evensidemargin}{0in}
\setlength{\textwidth}{6.5in}

\usepackage{palatino,graphics,amsmath,amssymb,enumitem}

\newcommand{\ds}{\displaystyle}
\newcommand{\vs}[1]{\vspace{#1in}}
\renewcommand{\vss}[1]{\vspace*{#1in}}
\newcommand{\bvec}{{\mathbf b}}
\newcommand{\cvec}{{\mathbf c}}
\newcommand{\dvec}{{\mathbf d}}
\newcommand{\evec}{{\mathbf e}}
\newcommand{\fvec}{{\mathbf f}}
\newcommand{\qvec}{{\mathbf q}}
\newcommand{\uvec}{{\mathbf u}}
\newcommand{\vvec}{{\mathbf v}}
\newcommand{\wvec}{{\mathbf w}}
\newcommand{\xvec}{{\mathbf x}}
\newcommand{\yvec}{{\mathbf y}}
\newcommand{\zvec}{{\mathbf y}}
\newcommand{\zerovec}{{\mathbf 0}}
\newcommand{\real}{{\mathbb R}}
\newcommand{\twovec}[2]{\left[\begin{array}{r}#1 \\ #2
    \end{array}\right]}
\newcommand{\ctwovec}[2]{\left[\begin{array}{c}#1 \\ #2
   \end{array}\right]}
\newcommand{\threevec}[3]{\left[\begin{array}{r}#1 \\ #2 \\ #3
  \end{array}\right]}
\newcommand{\cthreevec}[3]{\left[\begin{array}{c}#1 \\ #2 \\ #3
    \end{array}\right]}
\newcommand{\fourvec}[4]{\left[\begin{array}{r}#1 \\ #2 \\ #3 \\ #4
    \end{array}\right]}
\newcommand{\cfourvec}[4]{\left[\begin{array}{c}#1 \\ #2 \\ #3 \\ #4
    \end{array}\right]}
\newcommand{\mattwo}[4]{\left[\begin{array}{rr}#1 & #2 \\ #3 & #4 \\ \end{array}\right]}
\renewcommand{\span}[1]{\text{Span}\{#1\}}
\newcommand{\bcal}{{\cal B}}
\newcommand{\ccal}{{\cal C}}
\newcommand{\scal}{{\cal S}}
\newcommand{\wcal}{{\cal W}}
\newcommand{\ecal}{{\cal E}}
\newcommand{\coords}[2]{\left\{#1\right\}_{#2}}
\newcommand{\gray}[1]{\color{gray}{#1}}
\newcommand{\lgray}[1]{\color{lightgray}{#1}}
\newcommand{\rank}{\text{rank}}
\newcommand{\col}{\text{Col}}
\newcommand{\nul}{\text{Nul}}

\begin{document}

\noindent
{\bf Mathematics 227} \\ 
{\bf Matrix transformations}

\bigskip
\begin{enumerate}
\item Students in a school are sometimes absent due to illness.
  We will record the fractions of healthy and ill students on one day
  in a vector
  $\xvec = \twovec HI$
  where $H$ is the fraction of healthy students and $I$ is the
  fraction of ill students.  For instance, if
  $\xvec=\twovec{0.8}{0.2}$, 80\% of the students are healthy and 20\%
  of the students are ill.

  Suppose that
  \begin{itemize}
  \item 95\% of the students who are healthy one day are healthy the
    next.
  \item 50\% of the students who are ill one day are ill the next.
  \end{itemize}

  If we know the percentages of healthy and ill students one day, a
  matrix transformation $T:\real^2\to\real^2$ tells us the percentages
  the next day.

  \begin{enumerate}[label=(\alph*)]
  \item Find $T\left(\twovec10\right)$.  That is, if 100\% of the
    students are healthy and none are ill one day, what are the
    percentages the next day?

    \vs{1}
  \item Find $T\left(\twovec01\right)$.  That is, if 100\% of the
    students are healthy and none are ill one day, what are the
    percentages the next day?

    \vs{1}
  \item Use these results to find the matrix $A$ such that $T(\xvec) =
    A\xvec$.

    \vs{1}
    \newpage
  \item Suppose that 80\% of the students are healthy and 20\% are ill
    on Tuesday.  What are the percentages on Wednesday?

    \vs{1}
  \item What were the percentages on Monday?

    \vs{1}
  \item What will be the percentages on Thursday and Friday?

    \vs{1}
  \item You can study how the percentages evolve over a long time
    using Sage.  First, define the matrix $A$ and Tuesday's vector
    $\xvec$.  Then use the following piece of code to show how $\xvec$
    evolves over the next 20 days.

\begin{verbatim}
      for i in range(20):
          x = A*x
          print x

\end{verbatim}
    What happens to the fraction of healthy and ill students after a
    very long time?

    \vs{1}
        

  \end{enumerate}

  \item Matrix transformation perform geometric operations.  Go to
    {\tt gvsu.edu/s/0Jf} to study the effect that various matrix
    transformations have on the plane.  On the left is the plane
    before the transformation;  on the right is the plane after the
    transformation.

    Describe the geometric effect of the matrix transformations
    defined by

    \begin{enumerate}[label=(\alph*)]
    \item $A=\mattwo 2001$
      \vs{0.5}
    \item $A=\mattwo 2002$
      \vs{0.5}
    \item $A=\mattwo 0{-1}10$
      \vs{0.5}
    \item $A=\mattwo 1101$
      \vs{0.5}
    \item $A=\mattwo {-1}001$
      \vs{0.5}
    \item $A=\mattwo {1}000$
      \vs{0.5}
    \item $A=\mattwo 1001$
      \vs{0.5}
    \item $A=\mattwo {1}{-1}{-2}2$
      \vs{0.5}

    \end{enumerate}

    What matrix produces a $180^\circ$ rotation?

    \vs{1}
    What matrix produces a reflection over the line $y=-x$?

    \vs{1}

\end{enumerate}


\end{document}
