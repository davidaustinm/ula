\documentclass[12pt]{article}

\pagestyle{empty}
\setlength{\topmargin}{0in}
\setlength{\headheight}{0in}
\setlength{\topsep}{0in}
\setlength{\textheight}{9in}
\setlength{\oddsidemargin}{0in}
\setlength{\evensidemargin}{0in}
\setlength{\textwidth}{6.5in}

\usepackage{palatino,graphics,amsmath,amssymb,enumitem}

\newcommand{\ds}{\displaystyle}
\newcommand{\vs}[1]{\vspace{#1in}}
\renewcommand{\vss}[1]{\vspace*{#1in}}
\newcommand{\bvec}{{\mathbf b}}
\newcommand{\cvec}{{\mathbf c}}
\newcommand{\dvec}{{\mathbf d}}
\newcommand{\evec}{{\mathbf e}}
\newcommand{\fvec}{{\mathbf f}}
\newcommand{\qvec}{{\mathbf q}}
\newcommand{\uvec}{{\mathbf u}}
\newcommand{\vvec}{{\mathbf v}}
\newcommand{\wvec}{{\mathbf w}}
\newcommand{\xvec}{{\mathbf x}}
\newcommand{\yvec}{{\mathbf y}}
\newcommand{\zvec}{{\mathbf y}}
\newcommand{\zerovec}{{\mathbf 0}}
\newcommand{\real}{{\mathbb R}}
\newcommand{\twovec}[2]{\left[\begin{array}{r}#1 \\ #2
    \end{array}\right]}
\newcommand{\ctwovec}[2]{\left[\begin{array}{c}#1 \\ #2
   \end{array}\right]}
\newcommand{\threevec}[3]{\left[\begin{array}{r}#1 \\ #2 \\ #3
  \end{array}\right]}
\newcommand{\cthreevec}[3]{\left[\begin{array}{c}#1 \\ #2 \\ #3
    \end{array}\right]}
\newcommand{\fourvec}[4]{\left[\begin{array}{r}#1 \\ #2 \\ #3 \\ #4
    \end{array}\right]}
\newcommand{\cfourvec}[4]{\left[\begin{array}{c}#1 \\ #2 \\ #3 \\ #4
    \end{array}\right]}
\newcommand{\mattwo}[4]{\left[\begin{array}{rr}#1 \amp #2 \\ #3 \amp #4 \\ \end{array}\right]}
\renewcommand{\span}[1]{\text{Span}\{#1\}}
\newcommand{\bcal}{{\cal B}}
\newcommand{\ccal}{{\cal C}}
\newcommand{\scal}{{\cal S}}
\newcommand{\wcal}{{\cal W}}
\newcommand{\ecal}{{\cal E}}
\newcommand{\coords}[2]{\left\{#1\right\}_{#2}}
\newcommand{\gray}[1]{\color{gray}{#1}}
\newcommand{\lgray}[1]{\color{lightgray}{#1}}
\newcommand{\rank}{\text{rank}}
\newcommand{\col}{\text{Col}}
\newcommand{\nul}{\text{Nul}}

\begin{document}

\noindent
{\bf Mathematics 227} \\ 
{\bf Linear independence}

\bigskip
\begin{enumerate}
\item Consider the matrices 
  $$
  A = 
  \left[
    \begin{array}{cccc}
      3 & 0 & -1 & 1 \\
      1 & -1 & 3 & 7 \\
      3 & -2 & 1 & 5 \\
      -1 & 2 & 2 & 3 \\
    \end{array}
  \right],\hspace*{24pt}
  B = 
  \left[
    \begin{array}{cccc}
      3 & 0 & -1 & 4 \\
      1 & -1 & 3 & -1 \\
      3 & -2 & 1 & 3 \\
      -1 & 2 & 2 & 1 \\
    \end{array}
  \right].
  $$

  \begin{enumerate}[label=(\alph*)]
  \item  Are the columns of $A$ linearly independent?  Are the columns
    of $B$ linearly independent?   Explain your thinking.

    \vs{1.5}
  \item For the matrix above whose columns are linearly dependent,
    express one of the vectors as a linear combination of the others.

    \vs{1.5}
  \end{enumerate}

\item  Suppose that $\vvec_1$, $\vvec_2$, $\vvec_3$, and $\vvec_4$ are
  vectors in $\real^8$ and that $\vvec_2=\zerovec$.  What can you say
  about the linear dependence/independence of this set of
  four vectors in $\real^8$?

  \vs{1}
  If the vectors are linearly dependent, express one of them as a
  linear combination of the others.

  \vs{1}
  
\item  Explain why seven vectors in $\real^5$ cannot be linearly
  independent.

  \vs{1}
  What is the largest number of vectors in $\real^5$ that can be
  linearly independent?

  \vs{1}

  If you have three vectors in $\real^5$, can you guarantee that they
  are linearly independent?

  \vs{1}

\item Suppose that $\vvec_1, \vvec_2, \ldots, \vvec_n$ are vectors in
  $\real^{24}$ that are linearly independent and that span $\real^{24}$.
  What can you say about the pivot positions of the matrix $A$ whose
  columns are $\vvec_1, \vvec_2, \ldots, \vvec_n$?

  \vs{1.5}
  How many vectors are there in this set?

  \vs{1}
  Suppose that $\bvec$ is a vector in $\real^{24}$.  Explain how you
  know that $\bvec$ can be written as a linear combination of
  $\vvec_1,\vvec_2, \ldots,\vvec_n$.

  \vs{1}
  Suppose that $\bvec$ is a vector in $\real^{24}$.  Explain how you
  know that $\bvec$ can be written as a linear combination of
  $\vvec_1,\vvec_2, \ldots,\vvec_n$ in exactly one way.

  \vs{1}
  What can you say about the solution to the homogeneous equation
  $A\xvec = \zerovec$?
  
  \vs{1}
\item Suppose that $\vvec_1,\vvec_2,\ldots,\vvec_n$ are a set of
  vectors and that $\vvec_3$ is a scalar multiple of $\vvec_7$.  What
  can you say about the linear independence/dependence of this set of
  vectors?

  \vs{1}
  If $\vvec_1,\vvec_2,\ldots,\vvec_n$ is a linearly independent set of
  vectors, is it necessarily true that one vector is a scalar multiple
  of another one?

  \vs{1}
\item Suppose that $A$ is a matrix such that the equation
  $A\xvec=\bvec$ has exactly one solution for every vector $\bvec$.
  What can you say about the linear dependence/independence of the
  columns of $A$?

  \vs{1}
  What can you say about the span of the columns of $A$?


    

  

\end{enumerate}


\end{document}
