\documentclass[12pt]{article}

\pagestyle{empty}
\setlength{\topmargin}{0in}
\setlength{\headheight}{0in}
\setlength{\topsep}{0in}
\setlength{\textheight}{9in}
\setlength{\oddsidemargin}{0in}
\setlength{\evensidemargin}{0in}
\setlength{\textwidth}{6.5in}

\usepackage{palatino,graphics,amsmath,amssymb,enumitem}

\newcommand{\ds}{\displaystyle}
\newcommand{\vs}[1]{\vspace{#1in}}
\renewcommand{\vss}[1]{\vspace*{#1in}}
\newcommand{\bvec}{{\mathbf b}}
\newcommand{\cvec}{{\mathbf c}}
\newcommand{\dvec}{{\mathbf d}}
\newcommand{\evec}{{\mathbf e}}
\newcommand{\fvec}{{\mathbf f}}
\newcommand{\qvec}{{\mathbf q}}
\newcommand{\uvec}{{\mathbf u}}
\newcommand{\vvec}{{\mathbf v}}
\newcommand{\wvec}{{\mathbf w}}
\newcommand{\xvec}{{\mathbf x}}
\newcommand{\yvec}{{\mathbf y}}
\newcommand{\zvec}{{\mathbf y}}
\newcommand{\zerovec}{{\mathbf 0}}
\newcommand{\real}{{\mathbb R}}
\newcommand{\twovec}[2]{\left[\begin{array}{r}#1 \\ #2
    \end{array}\right]}
\newcommand{\ctwovec}[2]{\left[\begin{array}{c}#1 \\ #2
   \end{array}\right]}
\newcommand{\threevec}[3]{\left[\begin{array}{r}#1 \\ #2 \\ #3
  \end{array}\right]}
\newcommand{\cthreevec}[3]{\left[\begin{array}{c}#1 \\ #2 \\ #3
    \end{array}\right]}
\newcommand{\fourvec}[4]{\left[\begin{array}{r}#1 \\ #2 \\ #3 \\ #4
    \end{array}\right]}
\newcommand{\cfourvec}[4]{\left[\begin{array}{c}#1 \\ #2 \\ #3 \\ #4
    \end{array}\right]}
\newcommand{\mattwo}[4]{\left[\begin{array}{rr}#1 & #2 \\ #3 & #4 \\ \end{array}\right]}
\renewcommand{\span}[1]{\text{Span}\{#1\}}
\newcommand{\bcal}{{\cal B}}
\newcommand{\ccal}{{\cal C}}
\newcommand{\scal}{{\cal S}}
\newcommand{\wcal}{{\cal W}}
\newcommand{\ecal}{{\cal E}}
\newcommand{\coords}[2]{\left\{#1\right\}_{#2}}
\newcommand{\gray}[1]{\color{gray}{#1}}
\newcommand{\lgray}[1]{\color{lightgray}{#1}}
\newcommand{\rank}{\text{rank}}
\newcommand{\col}{\text{Col}}
\newcommand{\nul}{\text{Nul}}

\begin{document}

\noindent
{\bf Mathematics 227} \\ 
{\bf Determinants}

\bigskip
\begin{enumerate}
\item  Find the determinant of the matrix:
$$
A =
\left[
  \begin{array}{ccc}
    2 & 1 & -3\\
    1 & 4 & 0 \\
    -2& 4 & 1
  \end{array}
\right].
$$


\vs{1.5}
\item Find the determinant of the upper triangular matrix:
$$
U = \left[
  \begin{array}{ccc}
    3 & 1 & -3\\
    0 & -2 & 2 \\
    0& 0 & 5
  \end{array}
\right].
$$

\vs{1.5}
How can you quickly find the determinant of an upper triangular matrix?

\vs{1}
What is the determinant of the identity matrix?

\vs{1}
\newpage
\item Sage can easily find the determinant of a square matrix {\tt A}
  with either {\tt A.det()} or {\tt A.determinant()}. 
  Find the determinant of the matrix
  $$
  B = \left[
    \begin{array}{ccc}
      -1 & 4 & 0\\
      6 & 1 & 1 \\
      3 & 2 & 0 
    \end{array}
  \right].
  $$

  \vs{1}
\item We are interested in understanding a connection between the
  determinant of a matrix and the invertibility of that matrix.  To
  understand this connection, we will study the effect of the three
  row operations (interchange, 
  scaling, and row replacement) on determinants.  

  {\bf Interchange:} Take the matrix $B$
  above and interchange any two rows to obtain a matrix $B_1$.
  Compute $\det(B_1)$. How does it compare to $\det(B)$?  This is what 
  generally happens.

  \vs{1}
  If a matrix $A$ has a nonzero determinant and we interchange two
  rows, explain why the determinant of the new matrix is nonzero.

  \vs{1} {\bf Scaling:} Now scale the first row of $B$ by $3$ to
  obtain the matrix $B_2$.  Compute $\det(B_2)$ and compare it to
  $\det(B)$.  This is also what happens generally.

  \vs{1}
  If a matrix $A$ has a nonzero determinant and we scale a row by a
  nonzero number, explain why the determinant of the new matrix is
  nonzero.

  \vs{1}
  \newpage
  {\bf Row replacement:} Finally, perform a row replacement
  operation on $B$ to obtain $B_3$.  Compute $\det(B_3)$ and comapre
  it to $\det(B)$.  This is also what generally happens.

  \vs{1}
  If a matrix $A$ has a nonzero determinant and we perform a row
  replacement operation, explain why the determinant of the new matrix
  is nonzero.

  \vs{1}
  
\item If you have a matrix $A$ whose determinant is nonzero, what can
  you guarantee about the determinant of its row echelon form?
  Explain your thinking.

  \vs{1}
  Consider the following two $3\times3$ matrices, both of which are in
  reduced row echelon form:
  $$
  R_1 =
  \left[
    \begin{array}{ccc}
      1 & 0 & -3 \\
      0 & 1 & 4 \\
      0 & 0 & 0 \\
    \end{array}
  \right],\hspace*{24pt}
  R_2 =
  \left[
    \begin{array}{ccc}
      1 & 0 & 0 \\
      0 & 1 & 0 \\
      0 & 0 & 1 \\
    \end{array}
  \right].
  $$
  Find $\det(R_1)$ and $\det(R_2)$ (you shouldn't need Sage to do
  this).

  \vs{1}

  If $A$ is a $3\times3$ matrix with a nonzero determinant, which of
  these two matrices is possible as the
  reduced row echelon form of $A$?  Explain your thinking.

  \vs{1}
  What can you guarantee about the reduced row echelon form of $A$?

  \vs{1}
  \newpage
\item If $A$ a square matrix with a nonzero determinant, explain why $A$
  is invertible.
  
\end{enumerate}


\end{document}
